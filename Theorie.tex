\section{Theoretische Grundlagen}

\subsection{Wärmeequivalent}

\subsection{Molare Wärmekapazität}

\subsection{Wasserwert}

\subsection{Zwickelabgleich}

Der Zwickelabgleich erlaubt es einen idealen Mischungsverlauf mit $t=0$ anzunähern. Hierfür wird eine größere warme Wassermasse mit einer kleineren kalten Wassermasse gemischt. Damit nun der Wasserwert des Gefäßes bestimmt werden kann, werden vorher noch einige Messungen benötigt. Bevor der Mischversuch anfangen kann wird zuerst in beiden Gefäßen die Temperatur gemessen um herauszufinden wie viel wärme beide Gefäße an die Umgebung abgeben oder aufnehmen. Sobald dies geschehen ist wird nun gleichmäßig das warme Wasser in das kalte Wasser geschüttet, währenddessen wird das Mischwasser gerührt um eine homogene Vermischung zu gewährleisten. Gleichzeitig wird die Temperatur gemessen, sobald nun das Wasser komplett vermischt ist wird wieder die Temperatur gemessen um herauszufinden wie viel Energie an die Umgebung abgegeben wird.