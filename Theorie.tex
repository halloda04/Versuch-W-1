\section{Theoretische Grundlagen}

\subsection{Wärmeequivalent}
Da Wärme und mechanische Arbeit historisch getrennt waren, ist mit der Einführung von der Energie ein Umrechnungsfakor von Kalorien in Joule von nöten. Dieser Umrechnungsfaktor $x$ wird Wärmeäquivalent gennant. Das Wärmeäquivalent kann mit folgender Gleichung
\begin{equation}
    W_{el} = U~I~t
    \label{elektrische energei}
\end{equation}
bestimmt werden. Dabei $W_{el}$ die verrichtete Arbeit $U$ die Spannung $I$ der Strom und $t$ die Zeit.
\\
Wenn man so ein Gefäß mit Wasser erwärmt ist die aufgenommene Wärme Q gegeben durch
\begin{equation}
    Q = (mc_W + C)\Delta T
    \label{wärme energie}
\end{equation}

Dabei ist $m$ die Masse, $c_W$ die Wärmekapazität von Wasser, $C$ die Wärmekapazität des Gefäßes und $\Delta T$ die Temperaturänderung des Gefäßes. $C$ wird häufig durch den Wasserwert $w = \frac{C}{c_W}$ beschrieben; dieser gibt an, wie viel Wasser die gleiche Menge an Wärme aufnehmen kann wie das Gefäß.  
Mit dem Gleichverteilungssatz der statistischen Physik entfällt auf jeden Freiheitsgrad eine Energie von $\frac{1}{2} k_B T$. Somit trägt ein Mol eine Energie von
\begin{equation}
    E = \frac{f}{2}k_B~N_A \cdot T~,
\end{equation}
wobei $N_A$ die Avogadro-Konstante ist, welche zusammen mit der Boltzmann-Konstante die
allgemeine Gaskonstante $R = N_A~k_B$ ergibt. Das ergibt 
\begin{equation}
c_{mol} = \frac{f}{2}R
\end{equation}
Für komplexere Verbindungen besagt die Kopp-Neumann-Regel, dass die molare Wärmekapazität der gesamten Verbindung die Summe aus den Wärmekapazitäten der einzelnen Atome innerhalb der Verbindung ist. Diese Regelung trifft nicht vollständig zu und ist somit nur eine Näherung an die Wärmekapazität. Diese Regel stimmt vor allem bei sehr niedrigen Temperaturen nicht.


Der Wasserwert beschreibt die Masse an Wasser die die gleiche Menge an Energie aufnimmt wie das Kalorimeter des Veruches. Um diesen zu bestimmen wird
\begin{equation}
    \centering
\omega = \frac{m_{\text{W}}\cdot \left(T_\text{W}-T_\text{M}\right)}{T_\text{M}-T_\text{K}} - m_{\text{K}}
    \label{Wasserwert}
\end{equation}
als Formel verwendet hierbei kommen die Werte die nicht direkt gemessen werden kommen aus dem Zwickelabgleich.

\newpage


\subsection{Zwickelabgleich}
Mit dem Zwickelabgleich lässt sich ein Mischprozess, welcher naturgemäß Zeit benötigt, als instantan annähern; somit müssen keine Wärmeverluste an die Umgebung berücksichtigt werden. Dies funktioniert mit einer Extrapolation der Geraden der Temperatur des warmen sowie des kalten Wassers. Diese werden im ersten Schritt gemessen. Danach mischt man die zwei Flüssigkeiten und misst dabei kontinuierlich die Temperatur des Gemisches. Sobald dieses Gemisch seine maximale Temperatur erreicht hat, wird die Temperatur des Gemisches weiterhin gemessen und die daraus folgende Gerade ebenfalls extrapoliert. Visuell ist dies in Abbildung \ref{fg:zw} zu sehen.


\begin{figure}[H]
    \centering
    \includegraphics[scale = 0.7]{Bilder/Zwickel1.PNG}
    
    \caption{Graphische Darstellung der Messergebnisse für den Zwickelabgleich aus den drei Phasen: vor, nach und während dem Mischen}
    \label{fg:zw}
\end{figure}

In dieser Grafik wird nun derjenige Zeitpunkt $t^*$ gesucht, bei dem die Flächen A und B, welche in Abbildung \ref{fg:zw2} zu sehen sind, die gleiche Fläche besitzen.
 

\begin{figure}[H]
    \centering
    \includegraphics[scale = 0.7]{Bilder/Zwickel2.PNG}
    
    \caption{Exemplarische Auswertung der Messergebnisse nach der Methode des Zwickelabgleichs. Mit den Grau schraffierten Flächen A und B}
    \label{fg:zw2}
\end{figure}

Die senkrechte Linie durch den Zeitpunkt $t^*$ liefert dann an den Schnittstellen der extrapolierten Geraden die Punkte $T_W$, $T_M$ und $T_K$. Die Gleichheit der Flächen $A$ und $B$ lässt sich im Wesentlichen durch das Prinzip der Energieerhaltung erklären. Die der kalten Flüssigkeit zugeführte Wärme $Q^+$ wird für $t \to \infty$ vollständig an die Umgebung abgegeben und bleibt unverändert. Lediglich der zeitliche Verlauf des Wärmeaustauschs kann variieren. Der Prozess findet so lange statt, wie ein Temperaturunterschied $T_\mathrm{K} - T_\mathrm{A}$ zwischen dem Kalorimeter und der Umgebung besteht. Mithilfe dieser nun festgestellten Werte kann der Wasserwert des Gefäßes durch Formel \ref{Wasserwert} ermittelt werden.








