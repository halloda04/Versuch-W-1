\section{Theoretische Grundlagen}

\subsection{Wärmeequivalent}
Da Wärme und mechanische Arbeit historisch getrennt waren, ist mit der Einführung von der Energie ein Umrechnungsfakor von Kalorien in Joule von nöten. Dieser Umrechnungsfaktor $x$ wird Wärmeäquivalent gennant. Das Wärmeäquivalent kann mit folgender Gleichung
\begin{equation}
    W_{el} = U~I~t
    \label{elektrische energei}
\end{equation}
bestimmt werden. Dabei $W_{el}$ die verrichtete Arbeit $U$ die Spannung $I$ der Strom und $t$ die Zeit.
\\
Wenn man so ein Gefäß mit Wasser erwärmt ist die aufgenommene Wärme Q gegeben durch
\begin{equation}
    Q = (mc_W + C)\Delta T
    \label{wärme energie}
\end{equation}

Dabei ist $m$ die Masse, $c_W$ Die Wärmekapazität von Wasser und C die Wärmekapazität des Gefäßes und $\Delta T $ die Temperatur des Gefäßes. C wird häufig durch den Wasserwert $w = \frac{C}{c_W}$, dieser gibt an wieviel Wasser die selbe Menge an Wärme aufnehmen kann wie das Gefäß.
Mit dem Gleichverteilungsatz der Statistischen Physik, entfällt auf jeden Freiheitsgrad eine Energie
von $\frac{1}{2}k_B~T$. Somit trägt ein Mol eine Energie von 
\begin{equation}
    E = \frac{f}{2}k_B~N_A \cdot T~,
\end{equation}
wobei $N_A$ die Avogadro-Konstante ist, welche zusammen mit der Boltzmann-Konstante die
allgemeine Gaskonstante $R = N_A~k_B$ ergibt. Das ergibt 
\begin{equation}
c_{mol} = \frac{f}{2}R
\end{equation}

Der Wasserwert beschreibt die Masse an Wasser die die gleiche Menge an Energie aufnimmt wie das Kalorimeter des Veruches. Um diesen zu bestimmen wird
\begin{equation}
    \centering
\omega = \frac{m_{\text{W}}\cdot \left(T_\text{W}-T_\text{M}\right)}{T_\text{M}-T_\text{K}} - m_{\text{K}}
    \label{Wasserwert}
\end{equation}
als Formel verwendet hierbei kommen die Werte die nicht direkt gemessen werden kommen aus dem Zwickelabgleich.


\subsection{Zwickelabgleich}
Mit dem Zwickelabgleich lässt sich Mischprozess welcher Naturgemäß Zeit benötigt, als Instantan anähern, somit müssen keine Wärmeverluste an die Umgebung berücksichtigt werden. Dies Funktioniert mit einer Extrapolation der geraden der Tempeartur des warmen, sowie des kalten Wassers. Diese werden im ersten schritt gemessen. Danach mischt man die zwei Flüssigkeiten und misst dabei  kontinuierlich die Temperatur des Gemisches.Sobalt dieses Gemisch seine maximale Temperatur erreicht hat. Danach wird  die Temperatur des Gemisches weiterhin gemess und und die daraus folgende Gerade ebenfalls extarpoliert. Visuell ist das in Abbildung \ref{fg:zw} zu sehen.

\begin{figure}[H]
    \centering
    \includegraphics[scale = 0.7]{Bilder/Zwickel1.PNG}
    \label{fg:zw}
    \caption{Graphische Darstellung der Messergebnisse für den Zwickelabgleich aus den drei Phasen: vor, nach und während dem Mischen}
\end{figure}

In dieser Grafik wird nun der jenige Zeitpunkt $t^*$ gesucht bei dem die Flächen A und B, welche in Abbilldung \ref{fg:zw2}zu sehen sind, die Gleiche Fläche besitzen. 

\begin{figure}[H]
    \centering
    \includegraphics[scale = 0.7]{Bilder/Zwickel2.PNG}
    \label{fg:zw2}
    \caption{Exemplarische Auswertung der Messergebnisse nach der Methode des Zwickelabgleichs. Mit den Grau schraffierten Flächen A und B}
\end{figure}

Die senkrechte Linie durch den Zeitpunkt $t^*$, liefert dann an den Schnitstellen der extrapolierten Geraden die Punkte $T_W$, $T_M$ und $T_K$. 








