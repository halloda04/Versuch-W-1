\section{Versuchsbeschreibung}

\subsection{Versuchsaufbau}

\subsubsection{Zwickelabgleich}
Für den ersten Teil des Versuchs, den Zwickelabgleich, wird benötigt, eine Waage, zwei Thermometer, ein Becherglas, ein Kalorimeter, 120 gramm Warmes und 50 gramm kaltes Wasser, einen Rührer für das Kalorimeter und eine Stoppuhr.  

\subsubsection{Elektrisches Wärmeequivalent}
Für diesen Teil des Experiments wird benötigt, das Kalorimeter, 50 Gramm destiliertes Wasser, ein Thermometer, ein Netzgerät, zwei Multimeter, Strom Kabel zum Verbinden, einen Rührer für das Kalorimeter und die Stoppuhr.

\subsection{Versuchsdurchführung}

\subsubsection{Zwickelabgleich}

Um den Zwickelabgleich Durchzuführen wird nun das normale Becherglas mit den 50 gramm kalten Wasser befüllt und das Kalorimeter mit den 120 gramm warmen Wasser. In diesen zwei Gefäßen wird nun für 5 Minuten alle 30 Sekunden die Temperatur gemessen und festgehalten. Sobald diese 5 Minuten vorbei sind wird das kalte Wasser unter ständigem umrühren innerhalb von einer Minute in das Warme Wasser gemischt. Während dieser zeit wird im 10 Sekunden Abstand die Temperatur gemessen, sobald die Temperatur Spitze erreicht wird, wird nochmal eine Minute lang im 10 Sekunden Abstand gemessen. Danach wird erneut für 5 Minuten im 30 Sekunden Abstand gemessen. Diese Durchführung wird zweimal wiederholt, am besten mit unterschiedlichen Start Temperaturen des Warmen Wassers.

\subsubsection{Elektrisches Wärmeequivalent}

Schaltung Strom!!!

Für die Durchführung des zweiten Teils des Versuches, muss zuerst die Schaltung aus Bild (einfügen) nachgebaut werden und vom Betreuer kontrolliert werden. Nun kann das Kalorimeter mit dem destiliertem Wasser befüllt werden und geschlossen werden, daraufhin wird das Thermometer eingesetzt. Bevor die Stromschaltung eingeschalten wird, wird die Temperatur des Wassers für 5 Minuten alle 30 Sekunden gemessen. Sobald das geschehen ist kann das Netzgerät eingeschalten werden, welches vorher auf 12V und höchste Stromstärke eingestellt wurde. Während das Netzgerät das Kalorimeter aufheizt wird der Rührer betätigt und alle 30 Skeunden die Temperatur gemessen, dies wird gemacht bis sich die Temperatur des Wasser um 10 Kelvin erhöht hat. Daraufhin wird der Heizstrom wieder ausgeschalten und die Temperatur wird wieder für 5 Minuten alle 30 Sekunden lang gemessen und aufgezeichnet. 
