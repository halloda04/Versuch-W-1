\section{Versuchsbeschreibung}

\subsection{Versuchsaufbau}

\subsubsection{Zwickelabgleich}
Für den ersten Teil des Versuchs, den Zwickelabgleich, werden eine Waage, zwei Thermometer, ein Becherglas, ein Kalorimeter, 120 Gramm warmes und 50 Gramm kaltes Wasser, ein Rührer für das Kalorimeter sowie eine Stoppuhr benötigt.

\subsubsection{Elektrisches Wärmeequivalent}
Für diesen Teil des Experiments werden das Kalorimeter, 50 Gramm destilliertes Wasser, ein Thermometer, ein Netzgerät, zwei Multimeter, Stromkabel zum Verbinden, ein Rührer für das Kalorimeter sowie eine Stoppuhr benötigt.


\subsection{Versuchsdurchführung}

\subsubsection{Zwickelabgleich}

Um den Zwickelabgleich durchzuführen, wird nun das normale Becherglas mit 50 Gramm kaltem Wasser befüllt und das Kalorimeter mit 120 Gramm warmem Wasser. In diesen zwei Gefäßen wird nun für 5 Minuten alle 30 Sekunden die Temperatur gemessen und festgehalten. Sobald diese 5 Minuten vorbei sind, wird das kalte Wasser unter ständigem Umrühren innerhalb von einer Minute in das warme Wasser gemischt. Während dieser Zeit wird im 10-Sekunden-Abstand die Temperatur gemessen. Sobald die Temperaturspitze erreicht wird, wird nochmals eine Minute lang im 10-Sekunden-Abstand gemessen. Danach wird erneut für 5 Minuten im 30-Sekunden-Abstand gemessen. Diese Durchführung wird zweimal wiederholt, am besten mit unterschiedlichen Starttemperaturen des warmen Wassers.

\newpage

\subsubsection{Elektrisches Wärmeequivalent}

\begin{figure}[H]
    \centering
\includegraphics{Bilder/Stromkreis.png}
\caption{Der Schaltkreis für den zweiten Teil des Experiments \cite{W1}}
\label{Stromkreis}
\end{figure}

Für die Durchführung des zweiten Teils des Versuches muss zuerst die Schaltung aus Bild \ref{Stromkreis} nachgebaut und vom Betreuer kontrolliert werden. Nun kann das Kalorimeter mit dem destillierten Wasser befüllt und geschlossen werden, daraufhin wird das Thermometer eingesetzt. Bevor die Stromschaltung eingeschaltet wird, wird die Temperatur des Wassers für 5 Minuten alle 30 Sekunden gemessen. Sobald dies geschehen ist, kann das Netzgerät eingeschaltet werden, welches vorher auf 12\,V und die höchste Stromstärke eingestellt wurde. Während das Netzgerät das Kalorimeter aufheizt, wird der Rührer betätigt und alle 30 Sekunden die Temperatur gemessen. Dies wird so lange durchgeführt, bis sich die Temperatur des Wassers um 10 Kelvin erhöht hat. Daraufhin wird der Heizstrom wieder ausgeschaltet und die Temperatur wird erneut für 5 Minuten alle 30 Sekunden gemessen und aufgezeichnet.
 
