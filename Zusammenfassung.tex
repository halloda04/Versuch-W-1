\section{Zusammenfassung}

Das zentrale Ziel des Versuches, die Bestimmung des elektrischen Wärmeäquivalents, ist teilweise geglückt. Der bestimmte Wert von
\begin{equation}
    c_\mathrm{W} = 3{,}41 \,\frac{\mathrm{J}}{\mathrm{g\,K}}
\end{equation}
unterscheidet sich deutlich vom Literaturwert, ist aber dennoch in der gleichen Größenordnung. Gleiches gilt für die molare Wärmekapazität. Hier wurde ein Wert von 
\begin{equation}
    c_{\mathrm{W, mol}} = 61{,}43  \,\frac{\mathrm{J}}{\mathrm{mol\,K}}
\end{equation}
bestimmt. Die Messungen könnten auf mehreren Arten verbessert werden.  
Zum einen könnten während des Mischvorganges mehr Messungen genommen werden, um somit den Zeitpunkt $t^*$ besser zu bestimmen. Zudem könnten die zwei Flüssigkeiten gleichmäßiger vermischt werden, wozu z. B. ein Rohr verwendet werden könnte, durch welches ein konstanter Wasserfluss erzwingt wird. Des Weiteren könnte stärker gemischt werden, damit sich das warme Wasser gleichmäßiger verteilt. Versehentlich wurde auch eine Messreihe ohne Umrühren durchgeführt, was zu stark unterschiedlichen gemessenen Temperaturen führt. Des Weiteren sollte das Deriviergefäß nach jeder Messung auf den Zustand, den es beim Bestimmen des Wasserwertes hatte, gebracht werden.
