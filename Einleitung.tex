\section{Einleitung}

Der durchzuführende Versuch handelt vom elektrischen Wärmeäquivalent und behandelt somit ein grundlegendes physikalisches Prinzip. Dieses Prinzip wird überall dort angewendet, wo mithilfe elektrischer Energie Wärme erzeugt wird, wie zum Beispiel beim Kochen, in Heizgeräten oder in industriellen Prozessen. Der Versuch ist der Wärmelehre zuzuordnen, da hierbei die Umwandlung von elektrischer Energie in Wärme untersucht wird.  

Im ersten Teil des Versuches wird die Temperatur über einen längeren Zeitraum gemessen, um die zeitliche Entwicklung der Wärmeaufnahme durch das Wasser zu beobachten. Im zweiten Teil des Versuches wird ebenfalls über einen längeren Zeitraum die Temperatur gemessen, nun aber unter Verwendung einer elektrischen Heizspirale. Hierzu werden zusätzlich ein Netzteil und ein Multimeter benötigt, um die elektrische Leistung zu bestimmen und so das Wärmeäquivalent präzise zu berechnen.  

Historisch gesehen geht das Konzept des Wärmeäquivalents auf James Prescott Joule im 19. Jahrhundert zurück. Joule konnte experimentell nachweisen, dass Wärme und mechanische Arbeit äquivalent sind, und legte damit einen Grundstein für die moderne Thermodynamik. Seine Experimente zeigten erstmals quantitativ, dass Energie in verschiedenen Formen erhalten bleibt und umgewandelt werden kann, was das Fundament vieler heutiger Technologien bildet.