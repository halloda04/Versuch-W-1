\section{Einleitung}

Der Durchzuführende Versuch handelt von dem Elektrischen Wärmeequivalent und behandelt somit ein Grundlegendes Prinzip. Dieses Prinzip wird überall dort verwendet wo mithilfe von Elektrischer Energie etwas erhitzt wird. Dieser Versuch ist der Wärmelehre zuzuordnen, dadurch wird im ersten Teil des Versuches die Temperatur über einen längeren Zeitraum gemessen. Im zweiten Teil des Versuches wird wieder über einen längeren Zeitraum die Temperatur gemessen, da nun aber auch eine elektrische Heizspirale zum Einsatz kommt wird ein Netzteil und Multimeter zur Messung benötigt.