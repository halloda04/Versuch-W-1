\section{Auswertung}

\subsection{Zwickelabgleich}
Um den Wasserwert des Dewar-Gefäß zu ermitteln wird ein Zwickelabgleich durchgeführt.


Der Wasserwert \omega~wird durch die Formel
\begin{equation}
    \omega = \frac{m_W \cdot \left(T_W-T_m\right)}{T_m-T_K}-m_K
    \label{Wasserwert1}
\end{equation}
berechnet.\cite{Protokkoll} Hierbei beschreibt $m_W$ das Gewicht des warmen Wassers und $m_K$ das Gewicht des kalten Wassers. Aus Abbildung \ref{bi:zwickel60} bzw. \ref{bi:zwickel70} können dann mithilfe der extrapolierten Ausgleichsgeraden die Werte für die Temperaturen abgelesen werden. Diese sind in Tabelle \ref{tab:messwerte} zu sehen.



\begin{figure}[H]
    \centering
    \includegraphics[width=\linewidth, keepaspectratio]{Bilder/Zwicklerzuklein60.png}
    \caption{Grafische Auswertung (60°C) des Zwickelabgleichs zur Bestimmung des Wasserwerts des Kalorimeters. Die Ausgleichsgeraden wurden mittels Numpy.polyfit erstellt. Der Ideale Mischzeitpunkt wurde mittels Augenmaß gefunden.} 
    \label{bi:zwickel60}
\end{figure}

\begin{figure}[H]
    \centering
    \includegraphics[width=\linewidth, keepaspectratio]{Bilder/Zwicklerzuklein70.png}
    \caption{Grafische Auswertung (70°C) des Zwickelabgleichs zur Bestimmung des Wasserwerts des Kalorimeters. Die Ausgleichsgeraden wurden mittels Numpy.polyfit erstellt. Der Ideale Mischzeitpunkt wurde mittels Augenmaß gefunden.} 
    \label{bi:zwickel70}
\end{figure}

\begin{table}[H]
    \centering
    \caption{Messwerte der ersten und zweiten Durchführung des Zwickelabgleichs.}
    \label{tab:messwerte}
    \begin{tabular}{lcccc}
        \toprule
        Größe & Einheit & Teil 1 (60°C) & Teil 2 (70°C) \\
        \midrule
        Masse kalten Wassers $m_K$ & g & $51,0 \pm 0,1$ & $4,8 \pm 0,1$ \\
        Masse warmen Wassers $m_W$ & g & $120,8 \pm 0,1$ & $120,0 \pm 0,1$ \\
        Temperatur warmes Wasser $T_W$ & °C & 51,8 $\pm 1,5$°C & 56,4 $\pm 1,5$°C \\
        Temperatur kaltes Wasser $T_K$ & °C & 23,6 $\pm 1$°C & 24,5 $\pm 1$°C\\
        Mischtemperatur $T_m$ & °C & 40,7$\pm 1$°C & 43,7 $\pm 1$°C\\
        \bottomrule
    \end{tabular}
\end{table}

Beim der Ersten Durchführung wurde der idealisierte Mischzeitpunkt $t^*$ auf 15s geschätzt wobei diese schätzung recht ungenau ist, da zum einen die Größe der zwei Flächen geschtezt werden muss, was dadurch erschwert wird, dass es zu wenig Messpunkte in diesem Zeitraum gibt. Das sorgt dafür das nicht ganz klar ist wie die Flächen ausehen. Zum anderen sind die Ausgleichgeraden für die drei Temperaturen ebenfalls Fehlerbehaftet. Dies trägt enbenfalls dazu bei das die Flächen nicht wohl difeniert sind. Da sich dieser Zeitpunkt stark auf $T_W$, $T_W$ und $T_M$ auswirkt wurde hier ein recht hoher fehler von $\pm 1$°C angenommen. Für $T_W$ wurde sogar ein Fehler von $\pm 1$°C angenommen, da die ausgleichgerade hier die stärkste steigung hat und somit eine änderung von $t^*$ sich hier am meisten auswirkt. Der Fehler der Massen kommt von dem Ablesefehler der Feinwage.






\newpage
Grundsätzlich gilt für den Fehler des Wasserwerts 
\begin{equation}
\begin{aligned}
\Delta \omega = \pm \Big(
    &\left|\frac{\partial \omega}{\partial m_W}\right| \cdot \Delta m_W
    + \left|\frac{\partial \omega}{\partial m_K}\right| \cdot \Delta m_K
    + \left|\frac{\partial \omega}{\partial T_K}\right| \cdot \Delta T_K \\
    &+ \left|\frac{\partial \omega}{\partial T_W}\right| \cdot \Delta T_W
    + \left|\frac{\partial \omega}{\partial T_m}\right| \cdot \Delta T_m
\Big)
\end{aligned}
\end{equation}

Differenzieren und vereinfachen ergibt schließlich:
\begin{equation}
\begin{aligned}
\Delta \omega
= {} & \pm \Bigg[
    \Delta m_K
    + \frac{\Delta m_W (T_W - T_m) + \Delta T_W\, m_W}{T_m - T_K} \\
    &\quad
    + m_W
    \frac{\Delta T_m (T_W - T_K) + \Delta T_K (T_W - T_m)}
         {(T_m - T_K)^2}
\Bigg]
\label{Fehlerwasserwert}
\end{aligned}
\end{equation}



Daraus ergibt sich nun der Wasserwert des Dewar-Gefäßes für das 60°C warmes Wasser aus Gleichung \ref{Wasserwert} und \ref{Fehlerwasserwert}
\begin{equation}
(\omega_\mathrm{1} = 27,41 \pm 27) \text{g}.
\end{equation}

\begin{equation}
    (\omega_\mathrm{2} = 29,58 \pm 24) \text{g}.
\end{equation}

Die Zwei Wasserwerte sind jeweils in der Fehlerschranke enthalte. Diese ist jedoch unrealistisch hoch. Dies liegt and dem sehr konservativen Fehler für $T_W$, $T_W$ und $T_M$ von $\pm 1$°C bzw $\pm 1,5$°C. Trozdem ist wirkt das ergebniss realistisch da der relativer Fehler der zwei werte  $\frac{\omega_\mathrm{1}}{\omega_\mathrm{2}}$ mit 7,5 \% gering ist.




