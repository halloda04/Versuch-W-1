\section{Auswertung}

\subsection{Zwickelabgleich}
Um den Wasserwert des Dewar-Gefäß zu ermitteln wird ein Zwickelabgleich durchgeführt.
%\begin{figure}[H]
%    \centering
%    \includegraphics[width=\textwidth]{Bilder/AuswertungZwicklabgleich.jpeg}
%\caption{Eine graphische Auswertung des Zwickelabgleich}
%\label{Zwickelabgleich}
%\end{figure}
Der Wasserwert \omega~wird durch die Formel
\begin{equation}
    \omega = \frac{m_W \cdot \left(T_W-T_m\right)}{T_m-T_K}-m_K
    \label{Wasserwert}
\end{equation}
berechnet.\cite{Protokoll-W1} Hierbei beschreibt $m_W$ das Gewicht 
des warmen Wassers, $T_W$ die Temperatur des warmen Wassers, $T_\text{m}$ die idealisierte Mischtemperatur des warmen und des kalten Wassers, $T_\text{K}$ die Temperatur des kalten Wassers und $m_\text{K}$ die Masse des kalten Wassers. Die jeweils gemessenen Temperaturen konnten auf eine Genauigkeit von $\Delta T \pm 0{,}1\,^\circ\mathrm{C}$ bestimmt werden. Dadurch können die weiterführenden Geraden der Temperatur als konstante Geraden verwendet werden. Das Gewicht des warmen Wassers konnte auf eine Genauigkeit von $\pm 0{,}05\,\mathrm{g}$ bestimmt werden $\left( m_\text{K} = 51 \pm 0{,}1\,\mathrm{g} \right)$. Für den Messfehler beim kalten Wasser konnte der gleiche Messfehler erreicht werden somit ist die Masse des kalten Wassers durch $\left( m_\text{W} = 120,8 \pm 0{,}1\,\mathrm{g} \right)$ gegeben. Für den Fehler von $T_\text{W}$ und $T_\text{K}$ muss beachtet werden, dass der Zeitpunkt $t_0$ per Hand in das Diagramm eingefügt wurde und somit nicht den idealen Werten entspricht. Dadurch hat die Gerade $t_0$ eine geschätzte Genauigkeit von $\pm 2$ Sekunden, was bei einem geschätzten Wert von $t_0 = 35$ Sekunden zu $t_0 = 35 \pm 2$ Sekunden führt. Somit ergibt sich für $T_\text{W} = (einfügen)$, für $T_\text{K} = $ und für $T_\text{M} = $. Für die zweite Durchführung konnten alle Werte mit der selben Genauigkeit bestimt werden, draus folgt für $m_\text{W} = 120 \pm 0,1 g $, für $M_\text{K} = 49,8 \pm 0,1 g $, für $T_\text{W} = (einfügen)$, für $T_\text{K} = $ und für $T_\text{M} = $. 

\newpage
Grundsätzlich gilt für den Fehler des Wasserwerts 
\begin{equation}
    \Delta \omega = \pm \left( \left|\frac{\partial{\omega}}{\partial{m_W}}\right|\cdot \Delta m_W +
    \left|\frac{\partial{\omega}}{\partial{m_K}}\right|\cdot \Delta m_K +
    \left|\frac{\partial{\omega}}{\partial{T_K}}\right|\cdot \Delta T_K +
    \left|\frac{\partial{\omega}}{\partial{T_W}}\right|\cdot \Delta T_W +
    \left|\frac{\partial{\omega}}{\partial{T_m}}\right|\cdot \Delta T_m +
    \right).
\end{equation}
Differenzieren und vereinfachen ergibt schließlich:
\begin{equation}
    \Delta \omega = \pm \left( \Delta m_K + \frac{\Delta m_W \cdot \left(T_W - T_m\right)+ \Delta T_W \cdot m_W}{T_m - T_K}
+ m_W \cdot \frac{\Delta T_m \cdot \left(T_W - T_K\right) + \Delta T_K \cdot \left(T_W - T_m\right)}{\left( T_m - T_K\right)^2}
    \right)
    \label{Fehlerwasserwert}
\end{equation}

Daraus ergibt sich nun der Wasserwert des Dewar-Gefäßes aus Gleichung \ref{Wasserwert} und \ref{Fehlerwasserwert}
\begin{figure}[H]
    \centering
    $\omega$ = 24,84 \pm 2,12 Gramm
\end{figure}
Mögliche Fehlerquellen sind leicht verschobene Messzeitpunkte, da die 
Zeitabstände ziemlich gering sind und somit leicht Fehler unterlaufen 
können. Ungenauigkeiten beim Legen der idealen Mischgeraden zum Zeitpunkt 
$t_0$. Ein eindeutiger Fehler ist die verlorengegangene Flüssigkeit ganz 
am Anfang des Versuches.
