\documentclass[
    toc=bibliographynumbered,
    a4paper,
    11pt,
    twoside=false,]{article}


\usepackage[utf8]{inputenc}     
\usepackage[T1]{fontenc}        
\usepackage[ngerman]{babel}  
\usepackage{fontspec}   
\usepackage{geometry}   
\usepackage{titlesec}  
\usepackage{setspace}
\usepackage{fancyhdr} 
\usepackage{lipsum}
\usepackage{wrapfig}
\usepackage{csquotes}
\usepackage{siunitx}
\usepackage{amsmath}
\usepackage{float}
\setlength{\parindent}{0pt}
\usepackage[style=numeric-comp]{biblatex}
\addbibresource{literatur.bib}
\usepackage[
    colorlinks=false,   % keine farbigen Links
    pdfborder={0 0 0}   % keine Rahmen
]{hyperref}
\usepackage[labelfont=bf,textfont=normalfont]{caption}

\usepackage{graphicx}
\usepackage{booktabs} % Für professionelle Tabellenlinien
\usepackage{float}    % Für die [H]-Positionierung der Abbildungen

\sisetup{
    separate-uncertainty = true,
    multi-part-units = single
}

\fancyhf{}
\fancyfoot[C]{\thepage}
\thispagestyle{empty}
\fancyhead[L]{Gruppe 04}
\fancyhead[C]{\Versuchnummer {} \Versuch}
\fancyhead[R]{\Abgabedatum}

\geometry{left=3cm, right=3cm, top=3cm, bottom=3cm}

\usepackage{xcolor}             
\usepackage{graphicx}           


\usepackage{ragged2e}         
%\RaggedRight
\justifying


\newcommand{\Versuchnummer}{W1}                           %CHange iiit
\newcommand{\Versuch}{Elektrisches Wärmeequivalent}          %Change iiit
\newcommand{\Abgabedatum}{14.12.2025}                           %Change iiit
\newcommand{\Versuchsdatum}{03.12.2025}              %Change iiit






\newcommand{\sectionstyle}[1]{\color{teal!40!gray}\bfseries\LARGE #1}         
\newcommand{\subsectionstyle}[1]{\color{teal!40!gray}\bfseries\Large #1}
\newcommand{\subsubsectionstyle}[1]{\color{teal!40!gray}\bfseries\large #1}
\titleformat{\section}{\sectionstyle}{\thesection}{1em}{}
\titleformat{\subsection}{\subsectionstyle}{\thesubsection}{1em}{}
\titleformat{\subsubsection}{\subsubsectionstyle}{\thesubsubsection}{1em}{}











\begin{document}



\begin{titlepage}
    
    {\color{teal!40!gray}\fontsize{50}{30}\selectfont\bfseries Physikalisches\\ Anfängerpraktikum}\\
    \vspace{0,5cm}
    {\LARGE {\textbf {Universität Augsburg\\Wintersemester 2025/26}}\par}
    \vspace{3cm}
    

   {\LARGE \textbf {Versuch: {}\Versuchnummer {} \Versuch  }}\\ 
   \vspace{1cm}



   \begin{minipage}{0.5\textwidth}
\setstretch{1.8}

{\large Gruppe: \hfill G 04} % \hfill schiebt "G 04" nach rechts

\vspace{0.1cm}

{\large Versuchsdatum: \hfill \Versuchsdatum} % \hfill schiebt das Datum nach rechts

\vspace{0.1cm}

{\large Abgabedatum: \hfill \Abgabedatum} % \hfill schiebt das Datum nach rechts

\vspace{1.2cm}

{\large Gemeinsames Versuchsprotokoll}

\vspace{0.3cm}

{\large Ferdinand Frey\\Tom Glaser}

\end{minipage}

 %    \begin{minipage}{0.5\textwidth}
  %          \includegraphics[width =\linewidth]{Bilder/Logo.jpg}
   % \end{minipage}\hfill
     




\vspace{2cm}
     \begin{minipage}{0.5\textwidth}
        \centering
            \includegraphics[width =\linewidth]{Bilder/Logo.jpg}
    \end{minipage}\hfill
\end{titlepage}
\setcounter{page}{2}


\pagestyle{fancy}
\markboth{INHALTSVERZEICHNIS}{}
\tableofcontents
\newpage

\pagestyle{fancy}
\section{Einleitung}

Der durchzuführende Versuch handelt vom elektrischen Wärmeäquivalent und behandelt somit ein grundlegendes physikalisches Prinzip. Dieses Prinzip wird überall dort angewendet, wo mithilfe elektrischer Energie Wärme erzeugt wird, wie zum Beispiel beim Kochen, in Heizgeräten oder in industriellen Prozessen. Der Versuch ist der Wärmelehre zuzuordnen, da hierbei die Umwandlung von elektrischer Energie in Wärme untersucht wird.  

Im ersten Teil des Versuches wird die Temperatur über einen längeren Zeitraum gemessen, um die zeitliche Entwicklung der Wärmeaufnahme durch das Wasser zu beobachten. Im zweiten Teil des Versuches wird ebenfalls über einen längeren Zeitraum die Temperatur gemessen, nun aber unter Verwendung einer elektrischen Heizspirale. Hierzu werden zusätzlich ein Netzteil und ein Multimeter benötigt, um die elektrische Leistung zu bestimmen und so das Wärmeäquivalent präzise zu berechnen.  

Historisch gesehen geht das Konzept des Wärmeäquivalents auf James Prescott Joule im 19. Jahrhundert zurück. Joule konnte experimentell nachweisen, dass Wärme und mechanische Arbeit äquivalent sind, und legte damit einen Grundstein für die moderne Thermodynamik. Seine Experimente zeigten erstmals quantitativ, dass Energie in verschiedenen Formen erhalten bleibt und umgewandelt werden kann, was das Fundament vieler heutiger Technologien bildet.
\newpage
\section{Theoretische Grundlagen}

\subsection{Wärmeequivalent}

\subsection{Molare Wärmekapazität}

\subsection{Wasserwert}

\subsection{Zwickelabgleich}

Der Zwickelabgleich erlaubt es einen idealen Mischungsverlauf mit $t=0$ anzunähern. Hierfür wird eine größere warme Wassermasse mit einer kleineren kalten Wassermasse gemischt. Damit nun der Wasserwert des Gefäßes bestimmt werden kann, werden vorher noch einige Messungen benötigt. Bevor der Mischversuch anfangen kann wird zuerst in beiden Gefäßen die Temperatur gemessen um herauszufinden wie viel wärme beide Gefäße an die Umgebung abgeben oder aufnehmen. Sobald dies geschehen ist wird nun gleichmäßig das warme Wasser in das kalte Wasser geschüttet, währenddessen wird das Mischwasser gerührt um eine homogene Vermischung zu gewährleisten. Gleichzeitig wird die Temperatur gemessen, sobald nun das Wasser komplett vermischt ist wird wieder die Temperatur gemessen um herauszufinden wie viel Energie an die Umgebung abgegeben wird.
\newpage
\section{Versuchsbeschreibung}

\subsection{Versuchsaufbau}

\subsubsection{Zwickelabgleich}
Für den ersten Teil des Versuchs, den Zwickelabgleich, wird benötigt, eine Waage, zwei Thermometer, ein Becherglas, ein Kalorimeter, 120 gramm Warmes und 50 gramm kaltes Wasser, einen Rührer für das Kalorimeter und eine Stoppuhr.  

\subsubsection{Elektrisches Wärmeequivalent}
Für diesen Teil des Experiments wird benötigt, das Kalorimeter, 50 Gramm destiliertes Wasser, ein Thermometer, ein Netzgerät, zwei Multimeter, Strom Kabel zum Verbinden, einen Rührer für das Kalorimeter und die Stoppuhr.

\subsection{Versuchsdurchführung}

\subsubsection{Zwickelabgleich}

Um den Zwickelabgleich Durchzuführen wird nun das normale Becherglas mit den 50 gramm kalten Wasser befüllt und das Kalorimeter mit den 120 gramm warmen Wasser. In diesen zwei Gefäßen wird nun für 5 Minuten alle 30 Sekunden die Temperatur gemessen und festgehalten. Sobald diese 5 Minuten vorbei sind wird das kalte Wasser unter ständigem umrühren innerhalb von einer Minute in das Warme Wasser gemischt. Während dieser zeit wird im 10 Sekunden Abstand die Temperatur gemessen, sobald die Temperatur Spitze erreicht wird, wird nochmal eine Minute lang im 10 Sekunden Abstand gemessen. Danach wird erneut für 5 Minuten im 30 Sekunden Abstand gemessen. Diese Durchführung wird zweimal wiederholt, am besten mit unterschiedlichen Start Temperaturen des Warmen Wassers.

\subsubsection{Elektrisches Wärmeequivalent}

Schaltung Strom!!!

Für die Durchführung des zweiten Teils des Versuches, muss zuerst die Schaltung aus Bild (einfügen) nachgebaut werden und vom Betreuer kontrolliert werden. Nun kann das Kalorimeter mit dem destiliertem Wasser befüllt werden und geschlossen werden, daraufhin wird das Thermometer eingesetzt. Bevor die Stromschaltung eingeschalten wird, wird die Temperatur des Wassers für 5 Minuten alle 30 Sekunden gemessen. Sobald das geschehen ist kann das Netzgerät eingeschalten werden, welches vorher auf 12V und höchste Stromstärke eingestellt wurde. Während das Netzgerät das Kalorimeter aufheizt wird der Rührer betätigt und alle 30 Skeunden die Temperatur gemessen, dies wird gemacht bis sich die Temperatur des Wasser um 10 Kelvin erhöht hat. Daraufhin wird der Heizstrom wieder ausgeschalten und die Temperatur wird wieder für 5 Minuten alle 30 Sekunden lang gemessen und aufgezeichnet. 

\newpage
\section{Auswertung}

\subsection{Zwickelabgleich}
Um den Wasserwert des Dewar-Gefäß zu ermitteln wird ein Zwickelabgleich durchgeführt.
%\begin{figure}[H]
%    \centering
%    \includegraphics[width=\textwidth]{Bilder/AuswertungZwicklabgleich.jpeg}
%\caption{Eine graphische Auswertung des Zwickelabgleich}
%\label{Zwickelabgleich}
%\end{figure}
Der Wasserwert \omega~wird durch die Formel
\begin{equation}
    \omega = \frac{m_W \cdot \left(T_W-T_m\right)}{T_m-T_K}-m_K
    \label{Wasserwert}
\end{equation}
berechnet.\cite{Protokoll-W1} Hierbei beschreibt $m_W$ das Gewicht 
des warmen Wassers, $T_W$ die Temperatur des warmen Wassers, $T_\text{m}$ die idealisierte Mischtemperatur des warmen und des kalten Wassers, $T_\text{K}$ die Temperatur des kalten Wassers und $m_\text{K}$ die Masse des kalten Wassers. Die jeweils gemessenen Temperaturen konnten auf eine Genauigkeit von $\Delta T \pm 0{,}1\,^\circ\mathrm{C}$ bestimmt werden. Dadurch können die weiterführenden Geraden der Temperatur als konstante Geraden verwendet werden. Das Gewicht des warmen Wassers konnte auf eine Genauigkeit von $\pm 0{,}05\,\mathrm{g}$ bestimmt werden $\left( m_\text{K} = 51 \pm 0{,}1\,\mathrm{g} \right)$. Für den Messfehler beim kalten Wasser konnte der gleiche Messfehler erreicht werden somit ist die Masse des kalten Wassers durch $\left( m_\text{W} = 120,8 \pm 0{,}1\,\mathrm{g} \right)$ gegeben. Für den Fehler von $T_\text{W}$ und $T_\text{K}$ muss beachtet werden, dass der Zeitpunkt $t_0$ per Hand in das Diagramm eingefügt wurde und somit nicht den idealen Werten entspricht. Dadurch hat die Gerade $t_0$ eine geschätzte Genauigkeit von $\pm 2$ Sekunden, was bei einem geschätzten Wert von $t_0 = 35$ Sekunden zu $t_0 = 35 \pm 2$ Sekunden führt. Somit ergibt sich für $T_\text{W} = (einfügen)$, für $T_\text{K} = $ und für $T_\text{M} = $. Für die zweite Durchführung konnten alle Werte mit der selben Genauigkeit bestimt werden, draus folgt für $m_\text{W} = 120 \pm 0,1 g $, für $M_\text{K} = 49,8 \pm 0,1 g $, für $T_\text{W} = (einfügen)$, für $T_\text{K} = $ und für $T_\text{M} = $. 

\newpage
Grundsätzlich gilt für den Fehler des Wasserwerts 
\begin{equation}
    \Delta \omega = \pm \left( \left|\frac{\partial{\omega}}{\partial{m_W}}\right|\cdot \Delta m_W +
    \left|\frac{\partial{\omega}}{\partial{m_K}}\right|\cdot \Delta m_K +
    \left|\frac{\partial{\omega}}{\partial{T_K}}\right|\cdot \Delta T_K +
    \left|\frac{\partial{\omega}}{\partial{T_W}}\right|\cdot \Delta T_W +
    \left|\frac{\partial{\omega}}{\partial{T_m}}\right|\cdot \Delta T_m +
    \right).
\end{equation}
Differenzieren und vereinfachen ergibt schließlich:
\begin{equation}
    \Delta \omega = \pm \left( \Delta m_K + \frac{\Delta m_W \cdot \left(T_W - T_m\right)+ \Delta T_W \cdot m_W}{T_m - T_K}
+ m_W \cdot \frac{\Delta T_m \cdot \left(T_W - T_K\right) + \Delta T_K \cdot \left(T_W - T_m\right)}{\left( T_m - T_K\right)^2}
    \right)
    \label{Fehlerwasserwert}
\end{equation}

Daraus ergibt sich nun der Wasserwert des Dewar-Gefäßes aus Gleichung \ref{Wasserwert} und \ref{Fehlerwasserwert}
\begin{figure}[H]
    \centering
    $\omega$ = 24,84 \pm 2,12 Gramm
\end{figure}
Mögliche Fehlerquellen sind leicht verschobene Messzeitpunkte, da die 
Zeitabstände ziemlich gering sind und somit leicht Fehler unterlaufen 
können. Ungenauigkeiten beim Legen der idealen Mischgeraden zum Zeitpunkt 
$t_0$. Ein eindeutiger Fehler ist die verlorengegangene Flüssigkeit ganz 
am Anfang des Versuches.

\newpage
\section{Zusammenfassung}

Das zentrale Ziel des Versuches, die Bestimmung des elektrischen Wärmeäquivalents, ist teilweise geglückt. Der bestimmte Wert von
\begin{equation}
    c_\mathrm{W} = 3{,}41 \,\frac{\mathrm{J}}{\mathrm{g\,K}}
\end{equation}
unterscheidet sich deutlich vom Literaturwert, ist aber dennoch in der gleichen Größenordnung. Gleiches gilt für die molare Wärmekapazität. Hier wurde ein Wert von 
\begin{equation}
    c_{\mathrm{W, mol}} = 61{,}43  \,\frac{\mathrm{J}}{\mathrm{mol\,K}}
\end{equation}
bestimmt. Die Messungen könnten auf mehreren Arten verbessert werden.  
Zum einen könnten während des Mischvorganges mehr Messungen genommen werden, um somit den Zeitpunkt $t^*$ besser zu bestimmen. Zudem könnten die zwei Flüssigkeiten gleichmäßiger vermischt werden, wozu z. B. ein Rohr verwendet werden könnte, durch welches ein konstanter Wasserfluss erzwingt wird. Des Weiteren könnte stärker gemischt werden, damit sich das warme Wasser gleichmäßiger verteilt. Versehentlich wurde auch eine Messreihe ohne Umrühren durchgeführt, was zu stark unterschiedlichen gemessenen Temperaturen führt. Des Weiteren sollte das Deriviergefäß nach jeder Messung auf den Zustand, den es beim Bestimmen des Wasserwertes hatte, gebracht werden.

\newpage
\section{Anhang}

\begin{figure}[H]
    \centering
    \includegraphics[width=\linewidth, keepaspectratio]{Bilder/Unbase1nannt.jpg}
    
\end{figure}
\begin{figure}[H]
    \centering
    \includegraphics[width=\linewidth, keepaspectratio]{Bilder/Unbase221nannt.jpg}
    
\end{figure}
\begin{figure}[H]
    \centering
    \includegraphics[width=\linewidth, keepaspectratio]{Bilder/Unbasenannt.jpg}
    
\end{figure}
\newpage
\input{Literaturverzeichnis}



\end{document}
